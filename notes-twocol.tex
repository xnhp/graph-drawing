\documentclass[10pt,twocolumn]{article}


\usepackage[utf8]{inputenc}

\usepackage{lecture-notes-template/zsfgv}
\usepackage{lipsum}
\usepackage{multicol}

\begin{document}

\setcounter{section}{6}
\section{$st$-ordering \& incremental orthogonal drawing}

\paragraph{\ild{Topological ordering}} of a \textit{directed} graph is an
ordering so that for each $(u,v) \in E$, $u$ comes before $v$ in the ordering.

\paragraph{\ild{$st$-ordering}} is an ordering of the vertex set s.t. each $v_j$
for $2 \leq j \leq n-1$ has at least one neighbour $v_i$ with $i<j$ and a
neighbour $v_k$ with $k > j$.

\paragraph{\thm{Thm}} Let $G$ biconnected. Then, for given $s, t \in V$, an
$st$-ordering can be computed in linear time.
\proof{
  \begin{enumerate}
  \item Orient edges of $G$ to create an $s-t$-graph $G_{st}$ by calculating an
    oriented open ear decomposition.
  \item The topological numbering of $G_{st}$ is then an $st$-ordering.
  \end{enumerate}
}

\paragraph{\ild{Open ear decomposition}} $D=(P_0, ..., P_r)$ of an undirected
graph is a partition of $E$ into an ordered collection of edge-disjoint paths
s.t.
\begin{itemize}
\item $P_0$ is an edge
\item $P_0 \cup P_1$ is a simple cycle
\item both end vertices of $P_i$ are in $P_0 \cup ... \cup P_{i-1}$
\item no internal vertex of $P_i$ is in $P_0 \cup ... \cup P_{i-1}$
\end{itemize}
\textbf{Computation}: Perform a DFS. Ears will be ${s,t}$, and backwards edges and suffixes
of the spanning tree path leading up to the backwards edge. Orient edges like
other edge at ``source'' node of backwards edge.
%image: 07-7-15.png




\paragraph{\ila{alg} } (Incremental orthogonal drawing) Prerequisites: max
degree 4.


\setcounter{section}{7}
\section{Straightline drawings \& canonical ordering}


\paragraph{\ild{Canonical Ordering}} (cf. $st$-ordering) of a triangulated graph
is an ordering of the vertex set $V$, s.t.
\begin{itemize}
\item $v_1, v_2, v_n$ are on the outer face
\item for any $v_j$, $j \in \{3, ..., n-1\}$, it is adjacent to
  \begin{enumerate}
  \item at least two vertices with a lower index, \textcolor{gray}{and}
  \item at least one vertex with a higher index.
  \end{enumerate}
\end{itemize}
\textbf{Computation}: (\textit{``peeling order''}) For $k=n, ..., 3$, pick $v_k$ from
$\partial(G_k) \backslash \{v_1, v_2\}$ so that $v_k$ not incident to a chord in
$\partial(G_k)$ \iln{(because else if incident to cord, i.e. triangle with two edges in
  $\partial (G_k)$ $\rightarrow$ removing leaves only one edge in
  $\partial(G_k)$)}. \textbf{Existence}: minimal chord (extreme case: triangle) has at
least one vertex in $\partial(G_k)$ that is not part of a chord.

\paragraph{\thm{Thm.}} A canonical ordering of a triangulated planar graph can
be computed in linear time. \gray{i.e. can pick chord-free vertices in linear time}.
\proof{
  \begin{itemize}
  \item counter \textsc{Chords}$(v)$ for $v \in \partial(G_k)$
  \item list \textsc{Chordfree} to consider
  \end{itemize}
  Pop vertex from \textsc{Chordfree}, update counter of neighbours: total runtime $\bigO(\sum
  \mathit{deg}(v)) = \bigO(2m)$. Insert vertex into \textsc{Chordfree} if counter
  is 0.
}\\

Let $G_j$ be the graph induced by $\{v_1, ..., v_j\}$.

\paragraph{\thm{Prop.}} $v_{j+1}$ is on the outer face of $G_j$. \proof{ There
  ex. path $P$ from $v_{j+1}$ to $v_n$ whose internal vertices come after $v_{j+1}$
  in the ordering, in particular $P$ does not contain vertices of $G_j$! ---
  $v_n$ is on the outer face of $G$ and thus of $G_j$. --- Assume $v_{j+1}$ is
  in an inner face: In any embedding, $P$ has to cross the boundary of $G_j$,
  and this cannot happen in a vertex. Hence, we have an edge crossing,
  contradiction to planary of $G$.
}

\paragraph{\thm{Prop}} Edges connecting $v_j$ to $G_j$ are consecutive among the
edges connecting $G_j$ to $G \backslash G_j$ \note{(i.e. there is no edge in
  between that is incident to $w \in G \backslash G_j$)}. \proof{Follows from
  the above proof.}

% image: 07-3-4.png


\paragraph{\iln{Note}} When peeling, a vertex has at least two neighbours in the
remaining graph. %mentioned in lec09

\paragraph{\thm{Prop.}} If $G$ is triangulated, the boundary $\partial(G_j)$ is
a simple cycle. \proof{Induction.}

\paragraph{\ile{Applications}}
\begin{itemize}
\item Kandinsky Drawings
\item Straighline drawings for planar graphs
\end{itemize}

\subsection{Straightline drawings for planar graphs}

\textsc{Shift}-Algorithm of {Fraysseix-Pach-Pollack}.

\paragraph{\textit{Basic Idea}}
\begin{itemize}
\item Suffices to consider triangular graphs (else triangulate, layout, then
  remove added edges)
\item Construct drawing step-by-step by incrementally adding a single vertex and
  its incident edges, potentially shifting the rest of the graph s.t. no
  crossings appear.
\end{itemize}

\paragraph{\ila{Alg} } (\textsc{Shift}-Algorithm)
\begin{enumerate}
\item Begin with one triangle, place $v_1$ on $(0,0)$, $v_2$ on $(2,0)$ and $v_3$
  on $(1,1)$. \iln{(slopes on outer face always $-1$, $1$)}
\item According to canonical ordering, add a new vertex $v_k$. Let $v_l$ be its
  leftmost (first on $P$) adjacent neighbour, $v_r$ the rightmost. Add $v_k$ in
  center s.t. slopes on outer face are unity.
\item To avoid edge overlaps, move $v_l$, $v_r$ and all nodes left (resp. right)
  further away. \iln{(Consequently, $v_k$ will be placed higher up)}
\end{enumerate}

\begin{itemize}
\item Edges from $v_k$ to $w \in G_{k-1}$ can not create crossings with edges in
  $\partial(G_{k-1})$ \proof{Slope of $f \in \partial(G_{k-1}) \leq 1$ always
    and slope of other $e \geq 1$. Because $x$-coordinates are monotonous, there
  cannot be such an edge because there cannot be an endpoint at the implied position.}
\item $v_k$ is on a grid point if $x+y$ is even (because slope 1).
\item We have shown that newly added part does not introduce crossing. Now still
  have to show that shifting does not introduce new crossings on the inside,
  i.e. in $G_{k-1}$. We distinguish cases defined by the leftmost vertex to be shifted.
  \begin{itemize}
  \item For vertices $v \not= v_k$, it must necessarily be some $v \in
    \partial(G_{k-1})$ by construction of the algorithm (do use nodes that
    are adjacent to $v_k$ except for $v_l$ and $v_r$). Can directly apply IH.
  \item For $v = v_k$, using it as target for shift is the same as using the
    second-to-leftmost adjacent $v'$. But this is the same case as for $G_{k-1}$.
  \item Shifting edges incident to $v_k$ does not introduce crossings because
    they and their neighbourhood move rigidly with $v_k$.
  \end{itemize}
\end{itemize}

\paragraph{\thm{Prop.}} Manhattan distance between two vertices on
$\partial(G_k)$ is even. \proof{Distance remains or increases by two}

\paragraph{\thm{Prop.}} \textsc{Shift} runs in linear time.
\begin{itemize}
\item $y$-coordinates can be computed from canonical ordering.
\item for $x$-coordinates (of $v_k$) maintain a forest of \textit{rigid edges}
  (middle edges that never appear in the boundary) % image: 08-5-21.png
  \note{( such a tree is always shifted as one (per definition) )} and boundary
  edges of current $G_j$.
\end{itemize}


\section{Straighline Drawings, Schnyder-Woods}

\paragraph{\ild{Convex combination}} of points $p_1, ..., p_3$ is a linear
combination $v_1 p_1 + v_2 p_2 + v_3 p_3$ s.t. $v_1 + v_2 + v_3= 1$ \iln{(point
  lies in triangle)}.

\paragraph{\iln{Note}} In a triangulated graph, boundary of outer face is a triangle.

\paragraph{\ild{Barycentric representation}} of a graph is an injective map $v
\in V \mapsto (v_1, v_2, v_3) \in \rr^3$  with
\begin{itemize}
\item $v_1 + v_2 + v_3 = 1$ (convex combination)
\item for $\{u,v\} \in E$ and $w \in V \backslash \{u,v\}$ there ex. $k \in
  \{1,2,3\}$ with $u_k < w_k$ and $v_k < w_k$
\end{itemize}
It is \ild{weak} if
\begin{align*}
  (u_k, u_{k+1}) <_{\text{lex}} (w_k, w_{k+1}) ~\land~ (v_k, v_{k+1}) <_{\text{lex}} (w_k, w_{k+1}) ~~ \mod 3
\end{align*}
\iln{If two coefficients are equal, then the next one must be strictly smaller}

% image: 09-3-8.png

\paragraph{\thm{Thm}} Let $v \mapsto (v_1, v_2, v_3)$ be a weak barycentric
representation of $G$, $(p_1, p_2, p_3)$ a triangle. Then,
\begin{align*}
  v \in V \mapsto v_1 p_1 + v_2 p_3 + v_3 p_3
\end{align*}
yields a planar straightline drawing of $G$.

\paragraph{\ild{\textsc{SW}-Partition}} partition of the inner edges into three
disjoint oriented classes $T_1$, $T_2$, $T_n$ so that each is an
upwards-directed tree rooted at $v_1, v_2, v_n$ resp.; and have specific order
of edges around inner vertices as pictured. \textbf{Computation}: In a canonical
ordering, let the leftmost (outermost) edge be blue, rightmost green, point
downwards, others red pointing upwards. \textbf{Properties}:
\begin{itemize}
\item $T_1$ is rooted at 1, numbers from leaves to root decreasing
\item $T_2$ rooted at 2, numbers decreasing
\item $T_n$ numbers increasing
\end{itemize}
Indeed, such a partition also defines a canonical ordering (topological
numbering of acyclic graph defined by trees. )

% image: 09-5-21.png

Given an inner vertex $v$, we have a red/blue/green path to the resp. root.
These paths are disjoint \gray{ (involving red: increasing vs decreasing numbers;
others: by ordering of edges around a vertex) } --- These paths divide the graph
into three (closed, i.e. boundary also contained) regions (vertex sets).

% image: 09-6-4

\paragraph{\thm{Prop.}} The following mapping yields a barycentric representation
\begin{align*}
v_i := \frac{ |R_i(v) - P_{i-1}(v)|}{n-1}
\end{align*}
\iln{(subtract to not overcount vertices in boundaries, normalise s.t. convex
  combination)}


\paragraph{\thm{Cor}} Can compute straightline drawing in $(n-1) \times (n-1)$
grid in linear time (\textsc{Schnyder-Woods} algorithm).


\begin{enumerate}
\item Triangulate Graph
\item Draw within triangle $p_1=(0,0)$, $p_2=(n-1, 0)$, $p_3=(0, n-1)$
  \iln{(enables integer coordinates, cf. above)}
\item Compute SW-decomposition $T_1, T_2, T_3$
\item Draw an inner vertex $v$ at $(|R_2(v)| - |P_1(v)|,~ |R_3(v)| - |P_2(v)|)$
\end{enumerate}




\end{document}