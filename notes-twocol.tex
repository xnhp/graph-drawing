\documentclass[10pt,twocolumn]{article}


\usepackage[utf8]{inputenc}

\usepackage{zsfgv}
\usepackage{lipsum}
\usepackage{multicol}

\begin{document}

\part{Complexity Measures}

\section{Deterministic measures}

Let $T$ be an algorithm type. For an algorithm $A$ of type $T$, it must be
defined, on input $x$
\begin{itemize}
\item when $A$ terminates
\item what the result is
\end{itemize}

An algorithm $A$ of type $T$ \ild{computes} a mapping $\varphi_A: (\Sigma^*)^m
\rightarrow \Sigma^*$ with
\begin{align*}
  \varphi_A(x) :=
  \begin{cases}
    \text{result of $A$ on $x$} & \text{if $A$ terminates} \\
    \text{undefined} & \text{otherwise}
  \end{cases}
\end{align*} 

\paragraph{\ild{Def.}} A \ild{complexity measure} for $A$ of type $T$ is a mapping
\begin{align*}
  \Phi : \text{finite computation of $A$ of type $T$ on $x$} \mapsto r \in \nn
\end{align*}


\paragraph{\ile{Ex.}} examples for $\Phi$ are $T$-\textsc{DTIME}, $T$-\textsc{DSPACE}

\paragraph{\ild{Def.}} A \ild{complexity function} of $A$ of type $T$ is a
mapping $\Phi_A: (\Sigma^*)^M \rightarrow \nn$ with
\begin{align*}
  \Phi_A(x) :=
  \begin{cases}
    \Phi(\text{computation of $A$ on $x$})& \text{if $A$ terminates} \\
    \text{undefined} & \text{otherwise}
  \end{cases}
\end{align*} 

\paragraph{\ild{Def.}} \ild{worst-case complexity function}: $\Phi_A : \nn
\rightarrow \nn$ with
\begin{align*}
  \Phi_A(n) := \max_{\abs{x}=n} \Phi_A(x)
\end{align*}
\note{``worst'' complexity of all inputs of this length}

\paragraph{\ild{Def.}} Algorithm $A$ of type $T$ \ild{computes} $f$ \ild{in
  $\Phi$-complexity} $t$ iff
\begin{align*}
\text{
  $\varphi_A=f$ and $\Phi_A \leq_{ac} t$
  }
\end{align*}

\paragraph{\ile{Note}}
\begin{itemize}
\item If not specified, an ``algorithm'' is short for "algorithm $A$ of type
  $T$".
\item Alphabets are arbitrary but finite
\item Numbers are encoded dyadically
\end{itemize}

\paragraph{\ile{Ex.}} Some complexity classes of \textit{functions}:
\begin{itemize}
\item $F \Phi(t) := \{ f ~|~ \text{$f$ is total function and there ex. alg. $A$
    of type $T$ computing $f$ in $\Phi$-complexity $t$}\}$
\item $F \Phi( \mathcal{O}(t) ) := \bigcup_{k \geq 1} F \Phi (k \cdot t)$
\item $F \Phi( \text{Pol~} t ) := \bigcup_{k \geq 1} F \Phi(t^k)$
\end{itemize}

\paragraph{\ile{Ex.}} Complexity classes of \textit{languages}:
\begin{itemize}
\item $\Phi(t) := \{ L ~|~  \text{$L \subseteq \Sigma^*$ and there ex. alg.
    accepting $L$ in $\Phi$-complexity $t$} \}$
\item $\Phi(\mathcal{O}(t))$ similar to definition for functions
\item $\Phi( \text{Pol~} t)$ likewise
\end{itemize}

We now turn to complexity measures for Turing Machines.
\begin{itemize}
\item TMs are characterised by one read-only input tape, an arbitrary number of
  working tapes and a write-only output tape. The input tape can be
  \ild{one-way} infite or \ild{two-way} infinite.
\item The \ild{input} is the full initial configuration of input tape and all
  other tapes.
\item If a TM $M$ stops, the output is given by the leftmost consecutive word on
  the output tape.
\end{itemize}
\paragraph{\ild{Def}} Algorithm types for turing machines:

\begin{figure}[h]
  \centering
  \includegraphics[width=\linewidth]{img/1.1-algtypes}
\end{figure}

\paragraph{\ild{Def}} Let $M$ be a $T$-TM, let $\beta$ be a computation of
$M(x)$.
\begin{itemize}
\item \tdtime{$\beta$} $:=$ number of steps of $\beta$
\item \tdtime{$\beta$} $:=$ number of cells of write tape visited (or non-empty)
\end{itemize}
We can define complexity classes based on the characteristics of the employed
TM:
\begin{align*}
  (F)
  \left\{
  \begin{tabular}{@{}l@{}}
    0 \\ 1 \\ 2
  \end{tabular}
  \right\}
  -
  \left\{
  \begin{tabular}{@{}l@{}}
    $T$ \\ $kT$ \\ $\text{multi}T$
  \end{tabular}
  \right\}
  -
  \left\{
  \begin{tabular}{@{}l@{}}
    \text{\textsc{dtime}} \\
    \text{\textsc{dspace}}
  \end{tabular}
  \right\}
  (r)
\end{align*}

\paragraph{\ile{Ex.}}
\begin{itemize}
\item $\text{len}:x\mapsto \abs{x} \in F1-$\tdtime{$( 1+\varepsilon )n$} for all
  $\varepsilon > 0$, where $\abs{x}=n$
\end{itemize}





\end{document}